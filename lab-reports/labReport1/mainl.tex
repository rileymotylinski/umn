\documentclass{article}
\usepackage{amsmath}
\usepackage{gensymb}
\usepackage[makeroom]{cancel}
\usepackage{graphicx}
\usepackage{comment}
\usepackage{changepage}
\usepackage[margin=1in]{geometry}


\graphicspath{ {./images/} }

\begin{document}
\title{Acting Forces During Projectile Motion : An Analysis}

\author{Riley Motylinski}

\maketitle

\section{Abstract}
In this report, we sought to understand and confirm the universal forces acting on a projectile near the surface of the Earth
through observation of its motion via video analysis. We examined the forces by shooting a projectile at a 3 different angles and analyzing 
the position and velocity in both the x and y direction through Vernier video analysis. Ultimately we found $a_x = 0\pm0.36$ and $a_y \in \{-9.56\pm 1.46 m/s^2,-9.06\pm 1.34 m/s^2,-9.1\pm 0.1 m/s^2\}$. 
Additionally, the differences between our experimental and theoretical ranges were $\{0m,0.064m,0.17m\}$, suggesting significant inaccuracy in our prediction for $g$. From this
concluded that there were no forces acting in the x direction while remaining unable to absolutely prove that $a_y = g$

\section{Introduction}
Consider the common instance of an object flying through the air - perhaps a ball; Visually, we know that it moves in a parabolic path, but why? Why not in a straight line? It's important to understand projectile motion because 
it forms the basis for the movement of everything from a golf ball to entire planets.
So what exactly transforms and molds the shape of its arc? Is there a secret group of mole people pulling all of our projectiles down to keep us from realizing the ultimate truth that, in fact, gravity doesn't exist? Through controlled video analysis of a projectiles full range of motion in tandem with additional theoretical range calculations, we hope to understand 
the universal forces applied to said object near the surface of the Earth.

\section{Prediction}

The time it takes for an object to fall can be derived as such:
\begin{align*}
y_1 = y_0 + v_{0y}t + \frac{1}{2}a_yt^2 \\
0 = \Delta y + v_{0y}t + \frac{1}{2}a_yt^2
\end{align*}
which is a simple quadratic that can be solved via the quadratic equation:
\begin{equation}
    t = \frac{-v_{0y} \pm \sqrt{(v_{0y})^2 - 2a_y\Delta y}}{a_y}
\end{equation}

If we predict $a_x = 0$, $a_y = g \approx -9.81m/s^2$, then you can plug equation (1) into:
\begin{align*}
    x_1 = x_0 + v_{0x}t
\end{align*}

to get:

\begin{equation}
    \Delta x = R = v_{0x} \left(\frac{-v_{0y} \pm \sqrt{(v_{0y})^2 - 2g\Delta y}}{g}\right)
\end{equation}

Notice we can cancel then negative sign across the equation; in this case, this is the relevant answer. Additionally, it's essential to 
clarify what $\Delta y$ means in this context - that being the distance from the 0 or starting point to the max height of the arc.
Importantly, we have also chosen $g \approx -9.81 m/s^2$. Why? Well, if we were to draw a free body diagram for the projectile 
in the air, we would see that the only acting force would be the ball's weight or $mg$. By Newton's second law:

\begin{align*}
\Sigma F_y &= ma_y\\
\cancel{m}g &= \cancel{m}a_y\\
\Rightarrow a_y &= g 
\end{align*}

From here, using our aforementioned prediction for $g$ and given a set of initial measurements, we can use the equation to calculate 
the theoretical range of a projectile. These come out to be 0m, 1.02m, 1.37m for trails 1, 2, and 3, respectively. Whether our prediction of $g$ and $a_x$ is accurate rests upon the fact that our theoretical and 
experimental results for range align.

Furthermore, if the results of our experiment are accurately predicted by equation (1), then we can conclude the only force acting on the object
\textit{is} gravity, otherwise the forces would be unbalanced and the acceleration—whether in the x or y direction—would not be 0 and $g$, respectively.

\section{Procedure}
3 trials were conducted where a projectile was launched from a cannon. one vertical (straight up), one at 15$\degree$ from the vertical, and one at 25$\degree$ from the vertical. The launcher was angled
accordingly, then pushed in two "clicks".

The Ipod was positioned on a plane parallel to that of the motion of the ball and far enough away from the table such that it 
could capture the full range of motion. A meter stick was placed in frame for reference during video analysis.

Finally,  videos of the projectile in motion were analyzed with vernier video analysis, generating a discrete set of points which were used to find plots 
of x position vs. time as well as y velocity vs. time.
\section{Analysis}

\subsection{Trial 1 - Vertical Shot}

\begin{center}
   
    \includegraphics[width=0.49\textwidth]{VerticalYVel.png}
    \newline
    \textbf{Fig. 1a}
\end{center}

\subsection{Trial 2 - 15$\degree$ Shot}
\begin{center}
    \includegraphics[width=0.49\textwidth]{15YVel.png}
    \includegraphics[width=0.49\textwidth]{15XPos.png}
    
    \textbf{Fig. 2a} \hspace{200px} \textbf{Fig. 2b}
    

    
   


\end{center}
    
\subsection{Trial 3 - 25$\degree$ Shot}
\begin{center}
    \includegraphics[width=0.49\textwidth]{25YVel.png}
    \includegraphics[width=0.49\textwidth]{25XPos.png}
    \newline
    \textbf{Fig. 3a}\hspace{200px}\textbf{Fig. 3b}
\end{center}

\subsection{Additional Observations}
\textbf{Positional Uncertainty}: $\pm$ 0.36m. Measured blur of ball during video analysis. 
\newline
\textbf{Acceleration/X Velocity Uncertainty}: Trial 1 : 1.46 $m/s^2$, Trial 2 : 1.34 $m/s^2$, Trial 3 : 0.1 $m/s^2$. Formula used was $\|m_{extreme} - m_{regression}\|$


\subsection{Written Analysis}

First of all, uncertainties were obtained as mention above in section 5.4. Positional uncertainty (used in both x position and y velocity graphs) was found by measuring the maximum blur of the ball during the vernier video analysis step: 0.36m. The uncertainty for the acceleration was found by subtracting the slope of the extreme fit from the slope of the regression fit, then taking the absolute value.
This turned out to be $1.46 m/s^2$ for trial 1, $1.34 m/s^2$ in trial 2, and $0.1m/s^2$ for trial 3. Similarly, the x velocity's uncertainty was found the same way; the results were $0.03 m/s$ for trial 2 and $0.01 m/s$ for trial 3.

The experimental values for our range were $\pm$ 0.36m, 0.956 $\pm$ 0.36m, and  1.54 $\pm$ 0.36m for trials 1, 2, and 3, respectively. Subtracting these numbers from our predictive ranges, we find  the differences to be \{0m, 0.064m, 0.17m\}.
These numbers demonstrate a significant spread in our theoretical and experimental ranges, showing an inherent inaccuracy in our measurements.

From here a few conclusions can be made. First of all, $v_x$ is constant. Fig. 2b shows the velocity to be 1.15$\ m/s\ \pm$ 0.03, within our tolerance for uncertainty. The same holds true for Fig 3b., where $v_x = 2.02\ m/s\ \pm 0.01$. Therefore, the prediction $a_x = 0$ is correct.
On the other hand the slopes of our velocity vs. time graphs \{-9.56$\pm 1.46 m/s^2$,-9.06$\pm 1.34 m/s^2$,-9.1$\pm 0.1 m/s^2$\} provide a much greater spread for $a_y$, none of which exactly 
align with our prediction for $a_y = g$. So, while we cannot absolutely conclude that the only force in the y direction is gravity, accounting for the uncertainty in the slopes of our lines get us within $g$ for trials 1 and 2.


\subsection{Error Analysis}

First of all, the clamp attaching the launcher to the table was not  secured as well as it could have been. The launcher should have been exactly perpendicular 
to the surface such that the ball traveled in a straight line along it. In reality, the launcher could easily sway in between launches 
so that the ball was traveling at an angle relative to the surface. This would ultimately lead to inconsistent velocity measurement 
across the launch angles as the ball moves either towards the camera or away from it. 

Another possible source of error could have happened during the video analysis. The video analysis required tracking the motion of the ball via pacing 
points at each frame step, ultimately forming the full arc through a set of discrete points. Doing this by hand could lead to an irregularly traced path for 
the ball as it moves instead of the true parabolic path it follows. The inconsistency in the ball's tracked position could lead to variance in the true x and y velocity of the ball,
ultimately affecting the graphs and final calculations.

On a final note, Fig. 4 exemplifies the lack of consistency in our results. In trial 2 we overestimate the range and in trial 3 we underestimate the range. Had there been a pattern to the error such as the error always being negative or a always being off by a constant amount, there could possible be a systematic error in the way we were measuring, however such a big spread for differences means no such thing.
From this we conclude a not-insignificant error during the measurement step—both during video analysis and setup which was not systematic in nature. It would be fair to point out that to get equation 2 we assumed $a_x = 0$ and therefore \textit{that} could be the source of error (suppose $a_x \neq 0$?), however we have shown that $v_x$ is constant, therefore the inaccuracy in the calculations 
for the range have to be with our prediction of $g$

\subsection{Answering The Question}
From this, we cannot absolutely conclude that gravity is the only force in the y direction of a projectile. The model with which we predicted the motion of our projectile was not 
accurate enough to prove that the only force acting on a projectile is gravity. A possible improvement could be to account for air resistance as well as the shape of our projectile and the drag it leads to. Maybe a gust of wind from the HVAC system in the room? It was a fairly small projectile. We \textit{do} know however, that these forces would be exclusively in the y direction, as we have proven that acceleration is 0 in the x direction.
Just think about it, being able to accurately predict the path of a 
projectile would help reduce the threat of meteors!

\section{Conclusion}

In this report, we sought to understand and confirm the universal forces acting on a projectile near the surface of the Earth. We were able to confirm that there were no forces acting within
the x direction. However, after consistently measuring the acceleration of a projectile to be less than $g$ or with significant uncertainty (recall, $a_y \in \{-9.56\pm 1.46 m/s^2,-9.06\pm 1.34 m/s^2,-9.1\pm 0.1 m/s^2\}$) and calculating the differences between our experimental and theoretical ranges as \{0m, 0.064m, 0.17m\} , it was impossible to absolutely conclude whether gravity is the only force influencing
an objects range of motion along the y direction.

\end{document}
